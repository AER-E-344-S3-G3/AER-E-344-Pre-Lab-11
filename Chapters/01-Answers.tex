\chapter{Answers}
\label{cp:answers}
\section{Question 1}
\begin{importantbox}
    You should review and understand the concepts of flow patterns around an airfoil: attached flow, separation flow, airfoil stall, wake vortex flow.
\end{importantbox}

Attached flow refers to the flow of air over an airfoil, wing, or any other aerodynamic shape, where the flow remains smoothly attached to the surface without separation throughout the entire surface. In attached flow, the air particles have constant contact with the surface. This type of flow is desirable for efficient aerodynamic performance, as it minimizes drag and allows for predictable lift generation \citep{Anderson_2016}.

Separation flow occurs when the flow of a fluid, such as air, detaches from the surface of an object, usually an airfoil or wing, before reaching its trailing edge. This detachment of the flow can lead to a disruption in the aerodynamic properties of the object, resulting in decreased lift and increased drag. Separation flow is often undesirable in aerodynamics, as it can lead to loss of control and reduced efficiency of aircraft or other aerodynamic devices \citep{Anderson_2016}.

Airfoil stall refers to a sudden loss of lift and increase in drag experienced by an airfoil or wing when it exceeds its critical angle of attack. The critical angle of attack is the angle at which the flow over the airfoil becomes separated, leading to a decrease in lift-producing capabilities. When an airfoil stalls, it can lead to a loss of control and potentially dangerous flight conditions if not properly managed \citep{Anderson_2016}.

Wake vortex flow refers to the swirling vortices of air that are generated behind an aircraft as it moves through the atmosphere. These vortices are created as a result of the pressure differential between the upper and lower surfaces of the wings. Wake vortex flow is particularly significant during takeoff and landing, where aircraft are most susceptible to encountering the wake turbulence generated by preceding aircraft \citep{Anderson_2016}.

\section{Question 2}
\begin{importantbox}
    You should review and understand the concepts of Particle Image Velocimetry (PIV) technique and setup of a PIV system. 
\end{importantbox}

The PIV technique uses the displacement of particles in the flow between two points in time to determine a flow velocity field. 

The PIV set up system uses particle tracers to track the fluid movement across a wing, which are seen in an specific region using an illumination system made up of lasers and optics. A synchronizer is used for timing between a camera and illumination system, then the pictures are stored on a host computer \citep{lecture11}.